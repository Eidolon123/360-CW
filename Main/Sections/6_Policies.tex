\chapter{Policies}

\section{Purpose and Scope}
This section outlines the policies for employees and the IT team to enable safe and proper use of the Yotsuba Group network after relocation. The scope includes all company issued devices, network devices and external or remote devices that are connected to the network, in both the new and old headquarters.

\section{Work Issued Hardware}
Relocating office space has meant that some employees have been granted devices such as laptops and mobile phones. Hardware issued by Yotsuba Group should always be used solely for work purposes. These devices should not be left unattended and employees must configure multi-factor authentication. At a set time period of 6 months, employees must return their devices so that the IT team can perform inspections and security updates. This inspection will check for any signs of malware or suspicious activity.

\section{Backups}
Previously, there was an incident where Yotsuba Group lost parts of critical information on manufacturing designs for a new product. This was due to insufficient policies surrounding the management of stored data. It is now important that infrastructure is backed up weekly to reduce the risk of lost data in the future.

\section{Employee Onboarding/Offboarding}
When an employee joins or leaves Yotsuba Group, a thorough onboarding/offboarding process should be followed. As YG expands and gains exposure, there will be an increased chance of insider threat and targeted attacks. To minimise this, new employees will be enrolled on an independent online course that covers best practices for network security and internet safety.

For employees leaving YG, company issued devices should be returned, inspected and stored securely so that they can be issued to another employee. Additionally, their company IDs and accounts should be temporarily frozen to restrict external access. After a period of 3 months, these will then be permanently deleted in accordance with data protection guidelines.

\section{Remote Working}
The network design has accommodated for remote working. Because of this, employees working remotely must always use a designated VPN to ensure that data and communication between themselves and the network is encrypted. Failure to do this could result in the interception of sensitive data.

\section{Access Control}
Due to the new office building, it's important that only authorised employees have access. Network infrastructure should always be secured behind RFID locks, as the lack of physical security is what led to intellectual property theft at the previous headquarters.
