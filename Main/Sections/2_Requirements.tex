\chapter{Requirements and Assumptions}

\section{Expansion}
Yotsuba Group is a company experiencing rapid growth, hence the need for their new office space. It is assumed that this rapid expansion is to be around 10-20 (May need to be more scaleable) new employees per year. Because of this, there is a strong requirement for scalability, so the network can cope with this growth and there are no detrimental effects on network performance. It should be kept in mind that the selection of devices and routing protocols should support this need for scalability.
\section{Network Speeds and Bandwidth}
Research showed that private internet for the greater Tokyo region had available speeds in the range of 10Mbps to 1Gbps. It is assumed that enterprise internet speeds will be within a similar range and that the Yotsuba Group will be purchasing at the top range. Therefor a 10Gbps connection will be used for the designs.
\section{Employee breakdown}
As no information on individual department employee count was provided it has been assumed based on departmental needs.
\begin{itemize}
    \item Research and Technology - 50 employees
    \item Financial Planning - 20 employees
    \item Sales - 34 employees
    \item Material and Design - 50 employees
    \item Personnel - 10 employees
    \item Planning and Manufacturing - 60 employees
    \item Legal and Accounting - 10 employees
    \item Marketing - 20 employees
    \item IT - 16 employees
    \item Department Head and Assistants - 16 (8+8) employees
\end{itemize}
\section{Cisco in Japan}
The network will be using Cisco hardware, some of which will be transferred from the old 
building. Cisco (2021) press release demonstrates how the company plans to transition further into Japan through an agreement between the Japanese Government and Cisco on mass-scale digitalisation projects.
need to cite \parencite{cisco-japan}
\section{Physical Office Dimensions}
For floors U1, U2, G, 1, 2, 3, 4, 5, 6: 30mx50m - 64, 7m2 pp
Floor 7: 50mx20m - 16 \\
Floor7 Balcony: 50mx10m \\
The space provided to each employee workstation area was calculated via an online tool \parencite{floor-space}.
\section{Underground Carpark}
We are assuming that the two-floor underground car park does not currently have a good mobile signal and therefore, Wi-Fi APs could be implemented underground. This does depend on the budget of the organisation though because it is not necessarily something that is needed but would be helpful for employees who have parked underground as they can still make calls, send emails or do other work from their cars.
\section{Previous Devices}
\section{Extra Devices}